%-*-latex-*-
\documentclass[11pt]{article}
\newcommand{\blank}{\rule{2in}{0.4pt}}
\usepackage[numbers]{natbib}
\usepackage[letterpaper,margin=1in]{geometry}
%\usepackage{datetime}
%\renewcommand{\dateseparator}{-}
%\yyyymmdddate
%\usepackage{jeep}
%\cfoot{\thepage}
\usepackage{times,url,fullpage}
\newcommand{\heading}[1]{\bigskip\noindent\textbf{#1}~}
\setlength{\topmargin}{7mm}
\usepackage{xr}
\externaldocument[H-]{../homework/prob}
\externaldocument[H-]{../homework/drift}
\externaldocument[H-]{../homework/randommate}
\externaldocument[H-]{../homework/theta}
\externaldocument[H-]{../homework/neutraltheo}
\externaldocument[H-]{../homework/tree-b}
\externaldocument[H-]{../homework/seln}
\externaldocument[H-]{../homework/twoloc}
\externaldocument[H-]{../homework/inbreed}
\externaldocument[H-]{../homework/pstruc}
\externaldocument[H-]{../homework/admix}
\externaldocument[H-]{../homework/qchar}
\externaldocument[H-]{../homework/mmspec}
\begin{document}
\begin{flushleft}
Anthro 4950     \hfill  Generated \today\\
Prof: Alan Rogers\\
\end{flushleft}
\begin{center}
  \url{https://alanrogers.github.io/pgen/html/index.html}
\end{center}

\section*{\centering Human Evolutionary Genetics}

\heading{Response to COVID-19 pandemic} The University has cancelled
classes March 16--17, 2020. Beginning on Wednesday, March 18, this
will become an online class. There will be no further in-person
meetings of either the lecture section, the lab section, or the review
sessions.

To participate in an online class, you will need access to a
computer---preferably your own. If you don't have a computer, send
email to Alan Rogers. You'll also need access to the Zoom software,
which is available at no charge. You should download and install this
software on your computer. We'll use Zoom for review sessions and (now
optional) labs. All Zoom meetings will use the same URL:
\url{https://utah.zoom.us/j/4356401658}. To join a meeting, point your
browser or Zoom client at that URL.

Here's how we'll organize the course going forward:
\begin{itemize}
  \item Lectures will be recorded as videos (voice over slides) and
    placed on YouTube. You will find links to these videos on the
    class webpage (see URL above), not on Canvas. Look for the
    ``YouTube'' links.

  \item The lab assignments will henceforth be optional. Nonetheless,
    we encourage students to try the labs on their own, before the
    lab, which is still scheduled at 1PM Wednesday. The lab meeting
    will be via Zoom, as explained above. Students can ask questions,
    and I'll help with the projects. Because the labs are now
    optional, you don't need to hand them in.
    
  \item Review sessions will be scheduled during what used to be the
    lecture: Tue and Thu at 10:45~AM. Join these sessions using Zoom,
    as explained above.

  \item Homework assignments (rather than exams) will test knowledge
    of the algebraic theory. These should be handed in electronically,
    using Canvas. (I still need to set up assignments within Canvas to
    make this possible.) You can submit these in any format that we
    can read. One possibility is to do the homework on paper, and take
    a picture of this with your cell phone. Alternatively, you can
    type the answers in Word, LaTeX, or plain text.

  \item There will be no in-class exams. The second midterm is
    cancelled, and the final exam will be take-home and administered
    via Canvas. Rather than asking algebraic questions, we will
    emphasize general concepts, and each answer will be about a
    paragraph in length. We will write some sample questions during
    the next week.
\end{itemize}

\heading{Description} Theories and methods of molecular population
genetics, with emphasis on human examples. Using these tools, genetic
data can inform us about population history and adaptive
evolution. Laboratory exercises with the Python programming language
connect theory to data. Satisfies Quantitative Intensive Requirement.

\heading{Prerequisites} You should be comfortable with algebra and
first-semester calculus.  No prior knowledge of Python is
needed.

\heading{Grading} one midterm exam (17\%), a cumulative final (33\%),
weekly labs (25\% total), and weekly homeworks (25\% total).  Grades are
curved as explained on the website.

\heading{Extra credit} An extra credit assignment is available on the
course website. It provides practice in algebra and is available only
during the first half of the semester. The due date is listed in the
syllabus below.

\heading{Exams} The midterm is paper-and-pencil and takes place in the
lecture room. You may bring one 3X5 card containing handwritten notes
on both sides. You may bring a calculator, but do not load notes onto
the calculator. The final is take-home, administered via Canvas, and
open-book. 

\heading{Weekly computer lab} In this lab, students do projects using
the Python computer language. The lab assignments are short enough to
complete during the two-hour lab. The lab syllabus is available on the
class web site. The projects themselves are described in JEPy and in
the \emph{Lab Manual for Anth/Biol 5221}, which is also available on
the website. The lab is cancelled beginning Wed, March 18, 2020.

\heading{Homework} There are also paper-and-pencil homework
assignments, which are due at roughly weekly intervals as indicated in
the syllabus below. The homework assignments are available on the
class website. Answers to even-numbered problems are in the back of
the book of assignments. Only odd-numbered problems will be graded.

\heading{Required readings} are listed in the outline below and in
the list of references. The main text,
\begin{quote}
Gillespie, John. 2004.
\emph{Population Genetics, a Concise Guide}, 2nd edition
\end{quote}
is available at the bookstore. All other readings are on the class
website. In addition, we will occasionally assign other published
papers and notes of our own.  When we do, they will be available
either on paper or on the course web site.

\heading{Recommended readings} Hetland, Magnus
L. 2017. \emph{Beginning Python: From Novice to Professional}, 3rd
Edn.

%\heading{Discussion list} All students should enroll in the class
%email list, which is a place to ask questions about the course (and
%also to answer them).  We often use the list for important
%announcements involving review sessions, homework, and exams.  To
%enroll, point your browser at \url{http://lists.csbs.utah.edu}, and
%follow the link to \texttt{EvGen}. To post a question to the list, you
%must use the email account with which you enrolled in the list. Just
%send your question by email to \url{evgen@lists.csbs.utah.edu}.

\heading{Contact} For questions of general interest---the subject
matter of the course, what the exam will be like, etc.---please use
the class discussion list (see above). For private discussions, all of
us are available after class and by appointment. \emph{Rogers}: 4428
Gardner Commons, 801-581-5529, \url{rogers@anthro.utah.edu};
\emph{Seger}: 322 S Biology, 801-581-8478,
\url{seger@biology.utah.edu}; \emph{Cauceglia}
\url{j.cauceglia@gmail.com}; \emph{Weight}
\url{michael.weight@utah.edu}.

%\heading{Study sessions} We will staff two study sessions per week,
%both in Room 103B of Stewart Building (just North of Pioneer Memorial
%Theater). Emily DiBlasi will run a study session at 9~AM Monday and
%Ryan Bohlender will run one at 10~AM Thursday.

\heading{Learning outcomes}
\begin{description}
\item[Evolution] Students will be able to apply the principles of
  evolutionary theory to explain variation within species and change
  across time.

\item[Scientific reasoning] Students will be able to apply the process
  of science to identify knowledge gaps, formulate hypotheses, and
  test them against experimental and observational data to advance an
  understanding of the natural world.

\item[Quantitative reasoning] Students will be able to use
  mathematical and computational methods and tools to describe living
  systems and be able to apply quantitative approaches, such as
  statistics, quantitative analysis of dynamic systems, or
  mathematical modeling.
\end{description}

\heading{Equal access provisions} The University seeks to provide
equal access to its programs, services and activities for people with
disabilities.  If you will need accommodations in this class, then
reasonable prior notice must be given to the instructor and to the
Center for Disability Services, 162 Olpin Union. Call 581-5020 to
make arrangements.

\begin{tabbing}
\hspace*{.05\textwidth}\=20 Mix\=\hspace{.5\textwidth}\=RSi\=\kill
\textbf{Date}\>\>\textbf{Lecture}\>\>\textbf{Reading}\\
\input{syllabus}
Apr \> 23 H\> \textbf{Final exam} due by 11:59~PM\\
\end{tabbing}

\bibliographystyle{plainnat}
\bibliography{defs,molrec,arrpubs,arrunpub,arr,qchar}
\end{document}

\end{document}


